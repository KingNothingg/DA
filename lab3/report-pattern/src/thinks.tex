\section{Выводы}

Подводя итоги, хочется сказать про найденные мной недочёты. После профилировки я удивился, насколько долго работают функции сохранения и загрузки, и попытался ускорить их. Мне удалось добиться ускорения примерно в 5 раз, заменив работу с потоками на работу с файлами из С. Также, я заметил, что в функциях вставки, удаления и поиска неоптимально используется функция сравнения ключей. Она вызывалась по 3 раза последовательно, что на самом деле можно было заменить одним вызовом в начале функции и последующим сравнением с результатом. Также, в функции main находился страшный цикл:
\begin{lstlisting}[language=C++]
for(int i = 0; i < strlen(key); ++i) {
    key[i] = (char)tolower(key[i]);
}
\end{lstlisting}
Который был заменен на:
\begin{lstlisting}[language=C++]
keyLen = strlen(key);
for(int i = 0; i < keyLen; ++i) {
    key[i] = (char)tolower(key[i]);
}
\end{lstlisting}
Дело в том, что функция strlen вызывалась много раз, что приводило к значительному замедлению работы программы. Хочется сказать, что получить вердикт "OK" без использования valgrind'a и gprof'a сильно сложнее, чем с ними. 

\pagebreak
